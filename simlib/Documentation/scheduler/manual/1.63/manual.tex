\section{schedulerLibrary} 

\subsection{Scheduler Library Reference Manual}
The Scheduler Library is a collection of data structures and C-language procedures to implement a scheduler. 
Three different data structures (List, Heap and Calendar Queue) are also provided in order to grant the maximum flexibility and performance according to the number of the scheduled events.
This section describes:
\begin{itemize}
\item	The CalendarEvent, List, Heap, Calendar Queue and Auxiliary data structures;
\item	The procedures creating, populating, displaying and deleting members of the above data structures;
\item	Two examples of how this library can be used.

\end{itemize}

The Scheduler library is implemented in <em>C</em> language. It has been tested under \textit{Windows}, \textit{MacOS} and \textit{Ubuntu Linux} environments
using \textit{gcc} compiler.
This section include a few examples to show how the library can be deployed.

\paragraph{A note for the developers}
It is possible to enable a trace of the execution by setting the variable CE\_DEBUG to $1$
in the file called \textit{utilities.h} and recompiling again.
\subsubsection{Basic Concepts}


This section describes the basic concepts of the library.

\paragraph{Definitions}
The following is a list of definitions for of the data structures used by the library:

\begin{table}[]
\centering
\caption{Data Structures}
\label{datastructures}
\begin{tabular}{|l|m{9cm}|}
\hline
Item           & Description                                                                                     \\\hline
sCalendarEvent & This is the main data structure. It can be instantiated as a sList, a sHeap or a sCalendarQueue \\\hline
sHeap          & It corresponds to a heap data structure                                                         \\\hline
sList          & It corresponds to a double linked list data structure                                           \\\hline
sCalendarQueue & It corresponds to a CalendarQueue data structure                                                \\\hline
sParameters    & It corresponds to a set of parameters used by the procedures                                    \\\hline
sAuxlists      & It includes the pointers (first and last event) to an unsorted sList data structure \\\hline          
\end{tabular}
\end{table}

\paragraph{Data structures overview}
This section explains how the data structures used by the library are defined. For each of the below data structures, a C language variable type is available.
paragraph{List}
The \textit{sList} data structure consists of the following:
\begin{itemize}
\item A string \textit{id} representing a \textit{sList} identifier;
\item A double \textit{timestamp} ;
\item A \textit{next sList} pointer referring to the next node (set to NULL if the node is the last in the \textit{sList});
\item A \textit{previous sList} pointer referring to the previous node (set to NULL if the node is the first in the \textit{sList});
The following members are used by the Auxiliary list only:
\begin{itemize}
\item An integer \textit{W} representing a weight associated to the event.
\item An integer \textit{Threshold} representing the summation of the previous members' weights. This field is automatically updated every time there is an \textit{insert} operation from the \textit{head}
of the list.
\end{itemize}
\end{itemize}

\paragraph{Heap}

The \textit{sHeap} data structure consists of the following:
\begin{itemize}
\item	A string \textit{id} representing a Heap identifier;
\item	Aan array of pointers \textit{Index}. Each element of this array points to one array of timestamps referring to events. These arrays are allocated dinamically.
\item	An integer \textit{K} denoting the dimension of the \textit{Index} array;
\item 	An integer \textit{M} denoting the size of the array of doubles to which \textit{Index} cells point to.
\end{itemize}

\paragraph{Calendar Queue}
The \textit{sCalendarQueue} data structure consists of the following:
\begin{itemize}
\item	A string \textit{id} representing a Calendar Queue identifier;
\item	A double pointer of sList type called \textit{buckets}. Each element of this list is a list itself;
\item	An integer \textit{numberOfBuckets};
\item	An integer called \textit{bucketSize};
\item	A \textit{min }and a \textit{max} double arrays representing the lower and upper boundaries of CALENDAR QUEUE.
\end{itemize}

\paragraph{Calendar Event}
The \textit{Calendar Event}  data structure consists of the following:
\begin{itemize}	
\item A string id representing an a Calendar Event identifier;
\item	An integer \textit{currentChoice} representing one of the three data structures used (\textit{List}, \textit{Heap} or \textit{CalendarQueue});
\item	A double \textit{precision} to be used to compare double values;
\item A double \textit{howmany} referring to the current number of events included in the calendar event data structure;
\item	A union \textit{datastructures} including a \textit{sList}, a \textit{sHeap} and a \textit{CalendarQueue}. Only one of these data structures can be active at one time.
\end{itemize}	

The variable type for this data structure is called sCalendarEvent.

\paragraph{Parameters}
The \textit{sParameters}  data structure consists of the following:
\begin{itemize}
\item An integer \textit{choice} representing one of \textit{List}, \textit{Heap} or \textit{Calendar Queue};
\item A double \textit{epsilon}, used when comparing two doubles;
\item An integer \textit{numberOfBuckets} used to represent the number of \textit{Buckets} in a \textit{Calendar Queue};
\item An integer \textit{bucket\_width} used to represent the width of a \textit{bucket}in a \textit{Calendar Queue};
\item An integer \textit{howmany} used to denote the maximum number of events (i.e. timestamps);
\item An integer \textit{K} indicating the width of the \textit{Heap} memory array \textit{Index};
\item An integer \textit{M} indicating the number of slots associated to each member of the heap memory array.
\end{itemize}

\paragraph{Auxiliary Lists}
 This data structure includes an auxiliary list consisting of unsorted events, typically processed starting from the head or the tail of the list.
The \textit{AUXLISTS} data structure consists of the following:
\begin{itemize}
\item	A double \textit{N1} representing the total number of events in the first auxiliary list;
\item	A double pointer of sList type called \textit{auxF1} and \textit{auxL1} pointing respectively to the first and to the last event of the \textit{sList}.
\end{itemize}


\subsection{Library Algorithms overview}

This section provides a general introduction to the logic exploited by the algorithms creating, populating and destroying the \textit{Calendar Event}.

\paragraph{CALENDAR EVENT CREATION}
The \textit{CalendarEvent} data structure is created by the \textit{createCE} procedure. Since only one of \textit{List}, \textit{Heap} or \textit{CalendarQueue} data structure can be chosen, the following is performed:
\begin{itemize}
\item    It allocates the memory for the \textit{CalendarEvent} data structure.
\item	It sets the pointer to the \textit{List} to NULL or
\item	It initializes the \textit{Heap} data structure by using the \textit{initializeHeap} procedure or
\item	It initializes the \textit{Calendar Queue} data structure by calling the \textit{addCalendarQueue}  procedure.
\end{itemize}



\subsubsection{Populating the Calendar Event}

It is possible to add a new event (i.e. adding a new \textit{timestamp}) by using the addEvent procedure, which in turn calls one of \textit{addEventToList}, \textit{addEventToHeap} 
or \textit{addEventToCalendarQueue}.

\subsubsection{Deleteing or destroying the Calendar Events elements}

An event can be deleted (i.e. it is disconnected by the \textit{Calendar Event} and made available) or destroyed (the allocated memory for the event is freed). 
The same procedure is used for both the operations, by using a \textit{boolean} value (0/\textit{REMOVE} for the removal and 1/\textit{DESTROY} for the destruction)
 passed to the \textit{destroyEvent} procedure. Depending on the specific \textit{Calend Event} data structure,\textit{destroyEvent} calls in turn 
 \textit{delEventFromList}, \textit{delEventFromHeap} and \textit{delEventFromCalendarQueue}.

\subsubsection{Calendar Event destruction}

The \textit{Calendar Event} memory deallocation corresponds to two operations:
\begin{itemize}
\item	Deallocation of the specific data structure used;
\item	Deallocation of the Calendar Event  data structure as a whole.
\end{itemize}

This is achieved by calling the \textit{freeCalendarEvent} procedure, which in turns involkes one of \textit{freeList}, \textit{freeHeap} or \textit{freeCalendarQueue}.

\subsubsection{Pop Event}
	
 The \textit{pop} operation makes available the top (the smallest) event(i.e.timestamp) included in the \textit{Calendar Event}. According to the specific data structure, the following is performed by the \textit{popEvent} procedure:
 \begin{itemize}
\item	List: the first element is returned and the pointer to the first Event in the List is updated;
\item	Heap: the first element of the array is returned. The data structure is balanced if necessary (to preserve the Heap property);
\item	CalendarQueue: the popEventFromCalendarQueue procedure looks for the first non-empty bucket and it returns its first element. Then, the pointer to the first element of the bucket is updated.
\item    Auxiliary list (unsorted): in this case, it is possible to perform a pop operation in different ways: 
  \begin{itemize}
	\item Extract the first element of the list from the \textit{head} or from the \textit{tail} by using \textit{delItemAux} procedure. The extract policy is denoted by a flag with value \textit{HEAD} or \textit{TAIL};
	\item Extract an element from the list randomly with \textit{RPop} procedure by specifying a flag set to:
		\begin{itemize}
			\item \textit{UNIFORM}, meaning that the element is extracted randomly from the list (all the members of the list are equiprobable);
			\item \textit{WEIGHT}, meaning that the element is extracted randomly from the list taking an account a weight associated to it(the probability to extract a specific element of the list grows proportionally to the element's weight)
		\end{itemize}
  \end{itemize}

\end{itemize}


\subsubsection{PROCEDURES DESCRIPTION}
In the following, the procedures used are described in more detailed according to the data structure that uses them.
	
\paragraph{CALENDAR EVENT}


\subparagraph{createCE}

It creates a \textit{CalendarEvent} data structure, which can be one of \textit{List}, \textit{Heap} or \textit{CalendarQueue}. According to the user choice, the corresponding data structure is created (i.e. the memory is allocated).

Input: 
\begin{itemize}
\item A Parameter structure including the parameters for the CalendarEvent
\end{itemize}

Output:
\begin{itemize}
\item A pointer to the CalendarEvent data structure
\end{itemize}
sCalendarEvent * createCE(struct Parameters *pars )

\subparagraph{displayCE}
	 
It displays the content of the CalendarEvent data structure
Input
\begin{itemize}
\item A pointer to the CalendarEvent data structure
\end{itemize}
void displayCE(sCalendarEvent *pointer)

\subparagraph{addEvent}

It adds an event on the CalendarEvent data structure.

Input:
\begin{itemize}
 \item A pointer to the CalendarEvent;
 \item A pointer to a sList node including a timestamp
\end{itemize}
Output:
\begin{itemize}
\item An updated pointer to the sList node 
\end{itemize}
sList * addEvent(sCalendarEvent *ce, sList *)

\textit{Note}: it is possible to add a node that was previously removed (or extracted via a "pop" operation).
It is not possible adding a node that was early destroyed (in this case, a warning message is issued).

\subparagraph{popEvent}

It performs a \textit{pop} operation on the CalendarEvent (\textit{List}, \textit{Heap} or \textit{CalendarQueue}) data structure.

Input
\begin{itemize}
\item A pointer to a Calendar Event
\item A \textit{sList} double pointer (passage by reference) representing the extracted event, which includes the smallest timestamp in the \textit{CalendarEvent} 
\end{itemize}
void popEvent(sCalendarEvent *ce, sList **)


\subparagraph{freeCalendarEvent}

It releases all the data structures used (list, Heap or Calendar Queue)

Input: 
\begin{itemize}	
\item a pointer to the CalendarEvent structure
\end{itemize}
void freeCalendarEvent(sCalendarEvent *ce)

\subparagraph{createEventW}

It creates a sList node given a timestamp and a weight.
Input
\begin{itemize}
\item A timestamp
\end{itemize}
Output:
\begin{itemize}
\item A pointer to the new sList element.
\end{itemize}
sList * createEventW(double timestamp, int weight)

\subparagraph{freeEvent}

It performs a free() operation on a given sList pointer.

Input
\begin{itemize}
\item A double pointer (passage by reference) to sList data structure
\end{itemize}
void freeEvent(sList ** node)


\subparagraph{isCEEmpty}

It returns 1 (true) if the Calendar Event data structure is empty, 0 (true) otherwise.
Output
\begin{itemize}
\item A boolean value
\end{itemize}
boolean isCEEmpty(sCalendarEvent *ce)

\subparagraph{getCEFirst}

It returns the first event of the Calendar Event.

Input
\begin{itemize}
\item A pointer to a Calendar Event data structure
\end{itemize}

Output
\begin{itemize}
\item A pointer to a sList data structure
\end{itemize}
sList * getCEFirst(sCalendarEvent *ce)

\subparagraph{setEventTimestamp}

It sets the timestamp for an existing event. For example, you may perform a "pop" operation, set a new timestamp for the resulting event and add the event again to the
CalendarEvent data structure.

Input
\begin{itemize}
\item A pointer to a Calendar Event data structure
\item A double timestamp
\end{itemize}
Output:
\begin{itemize}
\item A pointer to a sList data structure including the new timestamp.
\end{itemize}
sList * setEventTimestamp(sList *node, double timestamp)




\subparagraph{getEventTimestamp}

It returns the timestamp value for a given pointer to sList data structure.
Input
\begin{itemize}
\item A pointer to a sList data structure
\end{itemize}
Output
\begin{itemize}
\item The double timestamp value.
\end{itemize}
double getEventTimestamp(sList * node)

\subparagraph{ceCurrentCapacity}
It returns the number of record in the \textit{Calendar Event} data structure.
Input:
\begin{itemize}
\item A pointer to the \textit{Calendar Event} data structure
\end{itemize}
Output:
\begin{itemize}
\item A double reprenting the number of events in the \textit{Calendar Event} data structure
\end{itemize}

double ceCurrentCapacity(sCalendarEvent *ce)


\subparagraph{createEvent}

It creates a sList node given a timestamp.
Input
\begin{itemize}
\item A timestamp
\end{itemize}
Output:
\begin{itemize}
\item A pointer to the new sList element.
\end{itemize}
sList * createEvent(double timestamp)

\subparagraph{destroyEvent}

It removes or destroys a sList node. A boolean parameter (set to \textit{REMOVE} or \textit{DESTROY}) allows to apply or not the \textit{free} function.
Input
\begin{itemize}
\item A pointer to a \textit{Calendar Event}
\item A pointer to a \textit{sList}
\end{itemize}
Output:
\begin{itemize}
\item A pointer to removed sList node.
\end{itemize}
sList * 	destroyEvent(sCalendarEvent *, sList *, boolean);

\paragraph{linearTimeTransform}

It applies a time linear factor \(\alpha t + \beta , where \alpha, \beta \in \Re, \alpha \geq 0\) to a \textit{Calendar Event}. 
 
Input:
\begin{itemize}
\item A pointer to the \textit{Calendar Event} 
\item A double \textit{alpha} representing \(\alpha\)
\item A double \textit{beta} representing \(\beta\)
\item A double \textit{precision} representing the precision used to compare double values (note that this is used for \textit{Calendar Queue} data structure)
\end{itemize}
void linearTimeTransform(sCalendarEvent *, double, double, double);


\paragraph{LIST}

\subparagraph{delEventFromList}

It destroys (or simply disconnects) a \textit{struct} from the \textit{List} or \textit{Bucket} (the same procedure is ued for both \textit{List} and \textit{Calendar Queue})
Input: 
\begin{itemize}
\item a pointer to the List/Bucket structure
\item a pointer to the node including the Event to be destroyed or disconnected
\item a flag specifying if the node has t be destroyed (with a "free" function) or just disconnected from the \textit{List} or \textit{Bucket}
\end{itemize}
sList * delEventFromList(sList *list , sList *node, boolean destroy)


\subparagraph{readList}

It prints the List (Bucket) content
Input
\begin{itemize}
\item The pointer to the first record;
\end{itemize}
void readList(sList *pointer)

\subparagraph{freeList}

It deallocates the memory for the list(bucket).

Input:
\begin{itemize}
\item A pointer to the list
\end{itemize}
void freeList(sList *pointer)

\subparagraph{addEventToList}

It adds an item to a \textit{List} or \textit{Bucket}, with respect to the order relationship (<)

Input:
\begin{itemize}
\item The pointer to the first element of the \textit{List} or \textit{Bucket}
\item The item to be added
\end{itemize}
Output:
\begin{itemize}
\item The updated pointer to the first item of the \textit{List} or \textit{Bucket}
\end{itemize}
sList * addEventToList(sList **first, sEvent event, double precision)

\subparagraph{linearTransformationList}

It applies a time linear factor \(\alpha t + \beta, where \alpha, \{beta \in \Re,
\alpha \geq 0\) to a \textit{List}. 
Input:
\begin{itemize}
\item A pointer to the \textit{List} 
\item A double \textit{alpha} representing \(\alpha;\)
\item A double \textit{beta} representing \(\beta;\)
\end{itemize}
void linearTransformationList(sList *, double, double);




\subsection{CALENDAR QUEUE}
\subparagraph{freeCalendarQueue}

It deallocates the memory for the CalendarQueue
Input:
\begin{itemize}
\item a pointer to the Calendar Queue
\end{itemize}
void freeCalendarQueue(sCalendarQueue *cq)

\subparagraph{addCalendarQueue}

\begin{itemize}
\item It creates the CalendarQueue record;
\item It creates numberOfBuckets buckets;
\item It initializes the min/max arrays, each one of bucketSize size
\end{itemize}
Input
\begin{itemize}
\item Pointer to the first element of the list;
\item The CalendarQueue identifier
\item The number of buckets
\item The Bucket size
\item The precision to compare double values
\end{itemize}

Output
\begin{itemize}
\item The pointer to the first record of the list/Bucket in the CalendarQueue
\end{itemize}
sCalendarQueue * addCalendarQueue(char *id, int numberOfBuckets, int bucketSize, double precision)

\subparagraph{addEventToCalendarQueue}

It adds an Event to the \textit{CalendarQueue} by choosing the right bucket.
Input:
\begin{itemize}
\item The pointer to the CalendarQueue record;
\item The item to be added
\end{itemize}
Output:
\begin{itemize}
\item A pointer to sList
\end{itemize}
sList * addEventToCalendarQueue(sCalendarQueue *calendarqueue, struct Event event, double precision)

\subparagraph{readCalendarQueue}

It prints the content of the CalendarQueue.
Input:
\begin{itemize}
\item The pointer to the first record of the CalendarQueue.
\end{itemize}
void readCalendarQueue(sCalendarQueue *pointer)

\subparagraph{DelEventFromCalendarQueue}

It deletes the first event (the smallest) from a \textit{Calendar Queue}. 
Input:
\begin{itemize}
\item A pointer to the \textit{Calendar Queue} 
\item A pointer to a \textit{sList} node representing the event to be deleted
\end{itemize}
void DelEventFromCalendarQueue(sCalendarQueue *, sList * );

\subparagraph{linearTransformationCalendarQueue}

It applies a time linear factor \(\alpha t + \beta , where \alpha, \beta \in \Re, \alpha \geq 0\) to a \textit{Calendar Event} to a \textit{Calendar Queue}. 
Input:
\begin{itemize}
\item A pointer to the \textit{Calendar Queue} 
\item A double \textit{alpha} representing \(\alpha\)
\item A double \textit{beta} representing \(\beta\)
\item A double \textit{precision} representing the precision used to compare double values
\end{itemize}
void linearTransformationCalendarQueue(sCalendarQueue *, double , double , double )

\paragraph{Heap}

\subparagraph{freeHeap}

It deallocates the memory reserved to the heap.
Input:
\begin{itemize}
\item A Pointer to the \textit{Heap}
\end{itemize}
void freeHeap(sHeap *heap)


\subparagraph{displayHeap}
It displays the \textit{Heap} content.
Input:
\begin{itemize}
\item A pointer to the \textit{Heap}
\end{itemize}
void displayHeap(sHeap *heap )


\subparagraph{initializeHeap}
It initializes the \textit{Heap}. Specifically, it allocates the memory for a \textit{sHeap} struct,
for the \textit{K Index} components and finally, it allocates the first array of timestamps pointed by \textit{Index[0]}.
Input:
\begin{itemize}
\item The \textit{K} and \textit{M} parameters
\end{itemize}
sHeap * initializeHeap(int K, int M)

\subparagraph{addEventToHeap}

It adds a timestamp to the Heap.
Input:
\begin{itemize}
\item A Pointer to the \textit{Heap}
\item A timestamp;
\item The requested precision to compare double values.>/li>
\end{itemize}
Output:
\begin{itemize}
\item A pointer to the \textit{sHEap} structure.
\end{itemize}
sHeap *addEventToHeap(sHeap *heap, double T, double precision)

\subparagraph{upHeap}
It balances the \textit{Heap} in case the \textit{heapmin} property has been violated by a call to \textit{addEventToHeap} procedure.
Input:
\begin{itemize}
\item The \textit{sList} pointer including the \textit{Heap} array. The usage of a \textit{sList} pointer  is requested to maintain coherence with
the insertion procedures belonging to the \textit{List} and \textit{CalendarQueue} data structures.
\item The array size;
\item The precision value needed to compare two doubles.
\end{itemize}
void upHeap(sList *a, int k, double precision)


\subparagraph{delEventFromHeap}
void delEventFromHeap(sHeap *heap, double precision, boolean destroy)

It deletes the first timestamp (the smallest) from the \textit{Heap}'s array. 
No \textit{free} operation is performed. The deleted element's position is recovered by the next \textit{addEventToHeap} procedure.
Input:
\begin{itemize}
\item A pointer to the \textit{Heap}\item 
\item The precision value requested to compare two doubles;
\item A boolean flag. Currently, it is ignored for the \textit{Heap}
\end{itemize}

\subparagraph{downheap}
It balances the \textit{Heap} in case the \textit{heapmin} property has been violated by a call to \textit{delEventFromHeap} procedure.
Input:
\begin{itemize}
\item The \textit{sList} pointer including the \textit{Heap} array. The usage of a \textit{sList} pointer  is requested to maintain coherence with
the insertion procedures belonging to the \textit{List} and \textit{CalendarQueue} data structures.
\item The position of the timestamp in the \textit{Heap}'s array to be deleted;
\item The precision value needed to compare two doubles.
\end{itemize}
void downheap(sList *a,int k,  int n, double precision)

\subparagraph{readHeap}
It reads the content of a specific Heap location.
Input:
\begin{itemize}
\item A pointer to the \textit{sHeap} data structure;
\item An integer denoting the element location
\end{itemize}
Output:
\begin{itemize}
\item The event timestamp corresponding to the location in input
\end{itemize}

double readHeap(sHeap *heap, int i)

\subparagraph{writeHeap}
It writes a timestamp event to a specific \textit{Heap} location. The memory is created dynamically in \textit{M} chunks (this parameter is specified by the user in the command line when
 the \textit{Scheduler} is invoked).
 Input:
 \begin{itemize}
 \item A \textit{sHeap} pointer;
 \item An integer representing the location where the timestamp is going to be written in the \textit{Heap};
 \item A double representing the actual event timestamp.
 \end{itemize}
 Output:
 \begin{itemize}
 \item An integer representing the success or the failure of the insertion.
 \end{itemize}
 int writeHeap(sHeap *heap, int i, double x)

\subparagraph{linearTransformationHeap}

It applies a time linear factor \(\alpha t + \beta , where \alpha, \beta \in \Re, \alpha \geq 0\) to a \textit{Heap}. 
Input:
\begin{itemize}
\item A pointer to the \textit{Heap} 
\item A double \textit{alpha} representing \(\alpha\)
\item A double \textit{beta} representing \(\beta\)
\end{itemize}
void linearTransformationHeap(Heap *, double , double )


\paragraph{Utilities}
\subparagraph{doubleCompare}

It compares two double variables.
Input:
\begin{itemize}
\item The two doubles \textit{a} and \textit{b} to be compared;
\item The precision \textit{epsilon}
\end{itemize}
Output:
\begin{itemize}
\item $0$  if a = b
\item $-1$ if a < b
\item $1$  if a > b
\end{itemize}
int doubleCompare(double a, double b, double epsilon)

\paragraph{AUXLISTS}

\subparagraph{addToAux}

It adds an sList * event to a \textit{sAuxlists}, without any ordering. The policy can be either FIFO or LIFO according to a flag (HEAD or TAIL) passed as parameter.
Input :
\begin{itemize}
\item The pointers to the first and last element;
\item The sList pointer to the event to be added
\item A flag set to HEAD or TAIL.
\end{itemize}
Output:
\begin{itemize}
\item The updated pointers to the first and last events in the \textit{sAuxlists}
\item An sList pointer to the added event.
\end{itemize}
sList * addToAux(sAuxlists *aux, sList * new, boolean mode)
\subparagraph{delItemAux}

It deletes the first or the last event to in a \textit{sAuxlists}. The policy can be either "delete from the head" or "delete from the tail" according to a flag (HEAD or TAIL) passed as parameter.

Input:
\begin{itemize} 
\item The pointers to the first and last element;
\item A flag set to HEAD or TAIL.
\end{itemize}
Output:
\begin{itemize}
\item An sList pointer to the added event.
\item The updated pointers to the first and last events in the \textit{sAuxlists}
\end{itemize}
sList * delItemAux(sList **, sList **  , boolean );

\subparagraph{createAUXLists}

It creates a \textit{sAuxlists} data structure, initializing also its members.


Output
\begin{itemize}
\item An sAuxlists pointer to the created \textit{sAuxlists} data structure
\end{itemize}

sAuxlists * createAUXLists()

\subparagraph{getHowmanyAux}

It returns the number of records in the unsorted \textit{sList} member of the \textit{sAuxlists} data structure.

Input
\begin{itemize}
\item An sAuxlists pointer to the \textit{sAuxlists} data structure
\end{itemize}

Output
\begin{itemize}
\item The current value of \textit{N1} member of \textit{sAuxlists} data structure
\end{itemize}

\subparagraph{freeAuxiliaryList}

It releases the \textit{sAuxlists} data structure memory allocation.

Input
\begin{itemize}
\item An sAuxlists pointer to the \textit{sAuxlists} data structure
\end{itemize}

\subparagraph{RPop}

It performs a \textit{pop} operation on a random basis, following two kind of policies: either the events are equiprobable (flag set to \textit{UNIFORM}) or depend on a weight (flag set to \textit{WEIGHT}).

It the extraction is performed on the weight, then the \textit{threshold} value for each node is re-calculated.

Input
\begin{itemize}
\item A \textit{sAuxlists} pointer to data structure memory;
\item A flag to denote the extraction policy, set either to  \textit{UNIFORM} or to \textit{WEIGHT}
\item A double pointer (passage by reference) to the extracted node.
\end{itemize}

void RPop(sAuxlists * aux, boolean flag, sList ** node)


\subsection{An example of Scheduler}
The Scheduler is an example of tool using the scheduler library. 

\paragraph{Usage}
The usage is:\\
\textit{scheduler} Usage: scheduler {<i/>data\_structure} [\textit{Index\_dimension} \textit{array\_dimension}] [\textit{number\_of\_buckets} \textit{bucket's width}]
Where:
\begin{itemize}
\item data\_structure is \textit{1} for LIST, \textit{2} for HEAP or \textit{3} for CALENDAR\_QUEUE
\item for HEAP, you must specify Index dimension and array(slot) dimension \item
\item for CALENDAR\_QUEUE, you must specify \textit{numberOfBuckets} and \textit{bucket\_width}
\end{itemize}

For example:
scheduler 1 
In this case, the scheduler has been invoked using a \textit{List}(\textit{1}).
scheduler 3 10 10
The scheduler has been invoked using a \textit{CalendarQueue}(3), using 10 buckets of size 10.
scheduler 2 4 10
The scheduler has been invoked using a \textit{Heap}(2), using an Index of dimension 4, each element of which is pointing to a slot of maximum 10 elements.

Two examples are here presented:
\begin{itemize}
\item A scheduler able to create events and store them on a \textit{List}, a \textit{Heap} or a \textit{Calendar Queue}
\item An example of the Auxiliary list usage.
\end{itemize}

\subsubsection{Using the Calendar Event data structure }


\subparagraph{Validation and variables declaration}
The Scheduler performs initially different steps:
\begin{itemize}
\item It declares a pointer to \textit{sCalendarEvent};
\item It performs a validation of the input parameters and load some embedded parameters (among which, for example, the precision \textit{epsilon}). It returns a pointer to the \textit{sParameters} structure;
\item It declares an array of \textit{sList} elements called \textit{nodes} used to input some timestamps and a further \textit{sList} element called \textit{out} that will be used later for a \textit{pop} operation. 
\end{itemize}

int main(int argc, char *argv[])\\
{\\
sCalendarEvent *ce;\\
sParameters *pars = validate(argc, argv);\\
sList *nodes[10], *out;\\
ce = createCE(pars);\\

The \textit{createCE} procedure is invoked passing the \textit{pars} parameters input. It returns a pointer to the \textit{CalendarEvent}, allocating the memory for the \textit{CalendarEvent} and 
the corresponding data structure (\textit{List}, \textit{Heap},
or \textit{CalendarQueue}).

\subparagraph{Adding events to the CalendarEvent}

It is necessary to invoke the \textit{addEvent} procedure, providing in input:
\begin{itemize}
\item A pointer to the CalendarEvent
\item a pointer to a sList structure including a timestamp (this can be achieved by using the \textit{createEvent} procedure).
\end{itemize}
addEvent returns a \textit{sList} pointer including the timestamp just added and - when using a \textit{List} or a \textit{CalendarList} - a valid \textit{next} and \textit{previous} pointers.
Next, it is convenient to print the \textit{CalendarEvent} content to verify that the \textit{addEvent} operation has completed successfully.

nodes[0] = addEvent(ce, createEvent(40.0)); 	displayCE(ce);\\

In the same fashion, it is possible to add more timestamps:

nodes[1] = addEvent(ce, createEvent(30.0));	displayCE(ce);\\
nodes[2] = addEvent(ce, createEvent(10.0));	displayCE(ce);\\

which adds other two timestamps (whose values are respectively 30 and 10) to the \textit{CalendarEvent}.

\subparagraph{Destroy, add and pop events}
The following operations performs the following:
\begin{itemize}
\item Destroy (a \textit{free} operation is performed since the \textit{DESTROY} flag is used) a timestamp from the \textit{CalendarEvent};
\item Add two new timestamps (with timestamps set respectively to 12 and 5);
\item Destroy (performing actually a \textit{remove} operation, that is without performing a \textit{free} operation since the \textit{REMOVE} flag is used) of \textit{nodes[3]}
\item Invoke the \textit{popEvent} procedure to extract the first (and smallest) timestamp;
\item Add another timestamp.
\end{itemize}

destroyEvent(ce, nodes[0], DESTROY);displayCE(ce);\\
nodes[3] = addEvent(ce, createEvent(12.0));	displayCE(ce);\\
nodes[4] = addEvent(ce, createEvent(5.0));	displayCE(ce);\\
destroyEvent(ce, nodes[3], REMOVE);displayCE(ce);\\
popEvent(ce, \&out);displayCE(ce);\\
nodes[5] = addEvent(ce, createEvent(1.0));displayCE(ce);\\


The nodes that were extracted (through a "pop" operation) or removed are re-inserted:

nodes[3] = addEvent(ce, nodes[3]);	displayCE(ce); \\
nodes[0] = addEvent(ce, nodes[0]);	displayCE(ce); \\
out = addEvent(ce, out);	displayCE(ce); \\

\subparagraph{Deallocating the memory}

Finally, the \textit{CalendarEvent} data structure is released. The \textit{nodes} and \textit{out} variables must released as well by the user.

freeCalendarEvent(ce);\\
}

\subparagraph{Using the Auxiliary list}

Sometimes, it is useful to store temporarily events that cannot be placed in the \textit{Calendar Event}. In this case, it is possible to use the Auxiliary list, which includes two pointers
to a sList data structure, respectively to the first and the last element. \textit{Pop} operations can be performed in different ways, either from the top or from the end of the list, or randomly - either
uniformly (all the events are equiprobable) or according to a weight associated to the event. The code performs the following:

\subparagraph{Initialization}

int demoDLL()\\
{\\
	sList *out1, *out2, *out3;\\\\\\


	out1 = (sList *)malloc(sizeof(sList));\\
	out2 = (sList *)malloc(sizeof(sList));\\
	out3 = (sList *)malloc(sizeof(sList));\\
\\
\\
	sList *node = NULL;\\
	sAuxlists * aux = createAUXLists();\\
 Three sList * variables \textit{out1}, \textit{out2} and \textit{out3} are created. The \textit{sAuxlists} data struture is istantiated.

\subparagraph{Adding the events}
Six new events are created. For each one, the \textit{timestamp} and the \textit{weight} are passed as arguments to \textit{createEventW} procedure. Correspondgly, the auxiliary list's content is displayed each time by 
using \textit{readListW} procedure.

	node = addToAux(aux, createEventW(12, 2), TAIL);readListW(aux->auxF1);\\
	node = addToAux(aux, createEventW(31, 3), TAIL);readListW(aux->auxF1);\\
	node = addToAux(aux, createEventW(7, 1), TAIL);readListW(aux->auxF1);\\
	node = addToAux(aux, createEventW(19, 4), HEAD);readListW(aux->auxF1);\\
	node = addToAux(aux, createEventW(150, 5), TAIL);readListW(aux->auxF1);\\
	node = addToAux(aux, createEventW(987, 2), HEAD);readListW(aux->auxF1);\\
	
\subparagraph{Removing the events by the tail or the head of the list}
Two elements are removed from the list, respectively the last (TAIL) and the first (HEAD). If the first element is removed, the \textit{delItemAux} re-calculates the thresholds for 
each component of the list.

	readEvent( delItemAux(aux, TAIL));printf("\\n");readListW(aux->auxF1);\\
	readEvent( delItemAux(aux, HEAD));printf("\\n");readListW(aux->auxF1);\\

\subparagraph{Random pop operation of the events by uniform or weighted policy}	
Three elements of the list are extracted by using a \textit{pop} operation exploiting either the \textit{weight} attribute (each element can be extracted according to a probability depending on the weight) or simply extracting randomly the event
 (being N the events in the list, the probability to extracted the \textit{i-th} is \textit{1/N}).
 
	RPop(aux, WEIGHT, \&out1);readListW(aux->auxF1);\\
	RPop(aux, WEIGHT, \&out2);readListW(aux->auxF1);\\
	RPop(aux, UNIFORM, \&out3);readListW(aux->auxF1);\\

\subparagraph{Release the memory}	
All the data structures are released.
	
	free(out1);\\
	free(out2);\\
	free(out3);\\
\\
	freeAuxiliaryList(aux);\\
\\
	return 1;